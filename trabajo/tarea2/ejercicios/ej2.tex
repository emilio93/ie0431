\renewcommand{\theenumi}{\large\bfseries\alph{enumi}}

\begin{ejercicio}
  \begin{enumerate}
    \item 
    Se obtiene el polinomio caracterítico del sistema.
    \begin{align*}
      p(s) = 1+C(s)P(s) &= 1+\frac{K_p(s+a)}{s(s+1)(s+2)(s+3)} = 0
      \\
      0 &= \frac{s(s+1)(s+2)(s+3)+K_p(s+a)}{s(s+1)(s+2)(s+3)}
      \\
      0 &= (s^2+s)(s^2+5s+6)+K_ps+K_pa
      \\
      0 &= s^4+5s^3+6s^2+s^3+5s^2+6s+K_ps+K_pa
      \\
      0 &= s^4+6s^3+11s^2+(6+K_p)s+K_pa
    \end{align*}

\renewcommand{\arraystretch}{1.5}
  \[
    \begin{matrix}
    \hline
    s^4 & 1 & 11 & K_p a
    \\\hline
    s^3 & 6 & 6+K_p & 0
    \\\hline
    s^2 & \frac{66-(6+K_p)}{6} & K_p a & 0
    \\
     & 66-6-K_p & 6K_p a & 0 & \bullet \times 6
    \\
     & 60-K_p & 6K_p a & 0
    \\\hline
    s^1 & \frac{(60-K_p)(6+K_p)-36K_p a}{60-K_p} & 0 & 0 
    \\
     & 360+54K_p-K_p^2 - 36K_p a & 0 & 0 & \bullet \times(60-K_p)
    \\
     & -K_p^2 + (54-36a)K_p + 360 & 0 & 0
    \\\hline
    s^0 & 6K_p a & 0 & 0
    \\
     & K_p a & 0 & 0 & \bullet / 6
    \\\hline
    \end{matrix}
  \]

  \begin{align*}
    K_p a > 0 \Rightarrow a > 0 \quad,\quad K_p > 0
    \\
    \\
    60-K_p > 0 \Rightarrow K_p < 60
    \\
    \\
    -K_p^2+(54-36a)K_p+360 > 0
    \\
    \frac{-K_p^2+54K_p+360}{36K_p} > a
  \end{align*}

  Con esto se obtienen los siguientes rangos para los cuales el sistema es estable.
  \begin{align*}
    0 < K_p < 60
    \\
    0 < a < \frac{-K_p^2+54K_p+360}{36K_p}
  \end{align*}

  Se comprueba una combinación de parámetros:

  \begin{center}
  \begin{tabular}{p{0.7\textwidth}}
  \toprule
    \begin{verbatim}
>> s=tf('s');
>> kp=30;
>> amax=(-kp^2+54*kp+360)/(36*kp)
amax =
    1
>> a=0.5;
>> C=Kp*(s+a)/(s+1);
>> P=1/(s*(s+2)*(s+3));
>> L=C*P;
>> sys=1/(1+L)
sys =
     s^4 + 6 s^3 + 11 s^2 + 6 s
  --------------------------------
  s^4 + 6 s^3 + 11 s^2 + 36 s + 15
Continuous-time transfer function.
>> p=pole(sys)
p =
  -5.1134 + 0.0000i
  -0.2094 + 2.4954i
  -0.2094 - 2.4954i
  -0.4678 + 0.0000i
    \end{verbatim}
  \bottomrule
  \end{tabular}
  \end{center}

  Todos los polos tienen parte real negativa, por lo que se trata de un sistema estable cuando $K_p=30$ y $a=0.5$.
  

    \item 
    \begin{enumerate}[I.]
      \item 
        \begin{align*}
          L(s) &= \frac{K_p}{s(s^2+s+1)(s+5)^2}
          \\
          p(s) &= s^5 + 11 s^4 + 36 s^3 + 35 s^2 + 25 s + K_p
        \end{align*}
        \[
          \begin{matrix}
          \hline
          s^5 & 1 & 36 & 25
          \\\hline
          s^4 & 11 & 35 & K_p
          \\\hline
          s^3 & \frac{361}{11} & \frac{275-K_p}{11} & 0
          \\
          & 361 & 275-K_p & 0 & \bullet \times 11
          \\\hline
          s^2 & \frac{9610-11K_p}{361} & K_p & 0
          \\
          & 9610-11K_p & 361 K_p & 0  & \bullet \times 361
          \\\hline
          s^1 & (9610-11K_p)(275-K_p) - 361^2K_p & 0 & 0
          \\
          & 2642750 - 136906 K_p - 11 K_p^2 & 0 & 0          
          \\
          & 240250 - 12446 K_p - K_p^2 & 0 & 0 & \bullet / 11
          \\\hline
          s^0 & 361K_p & 0 & 0
          \\\hline
          \end{matrix}
        \]

        \begin{align*}
          361K_p > 0 \Rightarrow K_p > 0
          \\\\
          240250 - 12446 K_p - K_p^2 > 0
          \\
          \Delta = (12446)^2 + 4\times 240250 = 155863916
          \\
          K_p = \frac{12446 \pm \sqrt{\Delta}}{-2} = -6223 \pm 361 \sqrt{299} \approx -12465.2735 \quad,\quad 19.2735
          \\
          \Rightarrow K_p > -12465-2735 \quad,\quad K_p < 19.2735
        \end{align*}
        El rango de $K_p$ para que el sistema sea estable es:
        \begin{align*}
          0 < K_p < 19.2735
        \end{align*}

      \item 
        \begin{align*}
          L(s) &= \frac{200}{s(s^3+6s^2+11s+6)}
          \\
          p(s) &= s^4 + 6s^3 + 11s^2 + 6s + 200
        \end{align*}
        \[
          \begin{matrix}
          \hline
          s^4 & 1 & 11 & 200
          \\\hline
          s^3 & 6 & 6 & 0
          \\
           & 1 & 1 & 0 & \bullet / 6
          \\\hline
          s^2 & 10 & 200 & 0
          \\
          & 1 & 20 & 0 & \bullet / 10
          \\\hline
          s^1 & -19 & 0 & 0
          \\
           & -1 & 0 & 0 & \bullet / 19
          \\\hline
          s^0 & -20 & 0 & 0
          \\\hline
          \end{matrix}
        \]

        Se tiene un cambio de signo, por lo que se tienen dos polos conjugados en el semiplano derecho del plano complejo. Se tiene un sistema inestable.


      \item 
        \begin{align*}
          L(s) &= \frac{128}{s(s^7+3s^6+10s^5+24s^4+48s^3+96s^2+128s+192)}
          \\
          p(s) &= s^8 + 3s^7 + 10s^6 + 24s^5 + 48s^4 + 96s^3 + 128s^2 + 192s + 128
        \end{align*}
        \[
          \begin{matrix}
          \hline
          s^8 & 1 & 10 & 48 & 128 & 128
          \\\hline
          s^7 & 3 & 24 & 96 & 192 & 0 
          \\
           & 1 & 8 & 32 & 64 & 0 & \bullet / 3
          \\\hline
          s^6 & 2 & 16 & 64 & 128 & 0 
          \\
          & 1 & 8 & 32 & 64 & 0 & \bullet / 2
          \\\hline
          s^5 & 0 & 0 & 0 & 0 & 0  & \text{Caso especial}
          \\
           & 6\times1 & 8\times4 & 32\times2 & 0 & 0 & \partial (s^6 + 8 s^4 + 32 s^2 + 64) / \partial s
          \\
           & 3 & 16 & 32 & 0 & 0& \bullet / 2
          \\\hline
          s^4 & {^8}/_3 & {^{64}}/_3 & 64 & 0 & 0
          \\
           & 1 & 8 & 24 & 0 & 0
          \\\hline
          s^3 & -8 & -40 & 0 & 0 & 0 
          \\
           & -1 & -5 & 0 & 0 & 0 & \bullet / 8
          \\\hline
          s^2 & 3 & 24 & 0 & 0 & 0 
          \\
           & 1 & 8 & 0 & 0 & 0 & \bullet / 3
          \\\hline
          s^1 & 3 & 0 & 0 & 0 & 0
          \\\hline
          s^0 & 8 & 0 & 0 & 0 & 0
          \\\hline
          \end{matrix}
        \]

        Se tienen dos cambios de signo, por lo que se tienen dos polos en el semiplano derecho del plano complejo. Se tiene un caso especial con todos los elementos de la fila 0, por lo que se tiene un par de polos en el eje imaginario. Se tiene un sistema inestable.
        
    \end{enumerate}
    \item
    \begin{enumerate}[I.]
    
      \item
        Se comprueba en matlab para 3 casos de $K_p$. Se muestra el caso inestable, el críticamente estable y el estable, en este orden.
        \begin{center}
        \begin{tabular}{p{0.7\textwidth}}
        \toprule
          \begin{verbatim}
>> s=tf('s');
>> Kp = 20;
>> L = Kp / (s * (s^2+s+1) * (s+5)^2 );
>> sys = 1 / (L+1)
sys =
    s^5 + 11 s^4 + 36 s^3 + 35 s^2 + 25 s
  ------------------------------------------
  s^5 + 11 s^4 + 36 s^3 + 35 s^2 + 25 s + 20
Continuous-time transfer function.
>> p=pole(sys)
p =
  -5.3881 + 0.0000i
  -4.4815 + 0.0000i
  -1.1505 + 0.0000i
  0.0101 + 0.8484i
  0.0101 - 0.8484i

>> Kp = 19.2736;
>> L = Kp / (s * (s^2+s+1) * (s+5)^2 );
>> sys = 1 / (L+1)
sys =
      s^5 + 11 s^4 + 36 s^3 + 35 s^2 + 25 s
  ---------------------------------------------
  s^5 + 11 s^4 + 36 s^3 + 35 s^2 + 25 s + 19.27
Continuous-time transfer function.
>>           p=pole(sys)
p =
  -5.3817 + 0.0000i
  -4.4931 + 0.0000i
  -1.1252 + 0.0000i
  0.0000 + 0.8417i
  0.0000 - 0.8417i

>> Kp = 19;
>> L = Kp / (s * (s^2+s+1) * (s+5)^2 );
>> sys = 1 / (L+1)
sys =
    s^5 + 11 s^4 + 36 s^3 + 35 s^2 + 25 s
  ------------------------------------------
  s^5 + 11 s^4 + 36 s^3 + 35 s^2 + 25 s + 19
Continuous-time transfer function.
>> p=pole(sys)
p =
  -5.3793 + 0.0000i
  -4.4975 + 0.0000i
  -1.1155 + 0.0000i
  -0.0039 + 0.8391i
  -0.0039 - 0.8391i
          \end{verbatim}
        \bottomrule
      \end{tabular}
      \end{center}

      \item
        Se comprueba dos polos en el semiplano derecho del plano complejo.
        \begin{center}
        \begin{tabular}{p{0.7\textwidth}}
        \toprule
          \begin{verbatim}
>> s=tf('s');
>> L = 200 / (s * (s^3 + 6*s^2 + 11*s + 6) );
>> sys = 1 / (L+1)
sys =
    s^4 + 6 s^3 + 11 s^2 + 6 s
  --------------------------------
  s^4 + 6 s^3 + 11 s^2 + 6 s + 200
Continuous-time transfer function.
>> p=pole(sys)
p =
  -4.2760 + 2.5409i
  -4.2760 - 2.5409i
  1.2760 + 2.5409i
  1.2760 - 2.5409i
          \end{verbatim}
        \bottomrule
      \end{tabular}
      \end{center}

      \item
      Se comprueban 2 polos inestables y dos polos críticamente estables.
      \begin{center}
        \begin{tabular}{p{0.7\textwidth}}
        \toprule
          \begin{verbatim}
>> s=tf('s');
>> L=128/(s*(s^7+3*s^6+10*s^5+24*s^4+48*s^3+96*s^2+128*s+192));
>> sys = 1 / (L+1)
sys =
     s^8 + 3 s^7 + 10 s^6 + 24 s^5 + 48 s^4 + 96 s^3 + ...
  ---------------------------------------------------- ...
     s^8 + 3 s^7 + 10 s^6 + 24 s^5 + 48 s^4 + 96 s^3 + ...
Continuous-time transfer function.
>> p=pole(sys)
p =
   1.0000 + 1.7321i
   1.0000 - 1.7321i
   0.0000 + 2.0000i
   0.0000 - 2.0000i
  -1.0000 + 1.7321i
  -1.0000 - 1.7321i
  -2.0000 + 0.0000i
  -1.0000 + 0.0000i
          \end{verbatim}
        \bottomrule
      \end{tabular}
      \end{center}
      
    \end{enumerate}

  \end{enumerate}
\end{ejercicio}