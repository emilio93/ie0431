\renewcommand{\theenumi}{\large\bfseries\alph{enumi}}

\begin{ejercicio}
  \begin{enumerate}
    \item 
    Se obtiene el polinomio caracterítico del sistema.
    \begin{align*}
      p(s) = 1+C(s)P(s) &= 1+\frac{K_p(s+a)}{s(s+1)(s+2)(s+3)} = 0
      \\
      0 &= \frac{s(s+1)(s+2)(s+3)+K_p(s+a)}{s(s+1)(s+2)(s+3)}
      \\
      0 &= (s^2+s)(s^2+5s+6)+K_ps+K_pa
      \\
      0 &= s^4+5s^3+6s^2+s^3+5s^2+6s+K_ps+K_pa
      \\
      0 &= s^4+6s^3+11s^2+(6+K_p)s+K_pa
    \end{align*}

\renewcommand{\arraystretch}{1.5}
  \[
    \begin{matrix}
    \hline
    s^4 & 1 & 11 & K_p a
    \\\hline
    s^3 & 6 & 6+K_p & 0
    \\\hline
    s^2 & \frac{66-(6+K_p)}{6} & K_p a & 0
    \\
     & 66-6-K_p & 6K_p a & 0
    \\
     & 60-K_p & 6K_p a & 0
    \\\hline
    s^1 & \frac{(60-K_p)(6+kp)-36K_p a}{60-K_p} & 0 & 0 
    \\
     & 360+54K_p-K_p^2 - 36K_p a & 0 & 0
    \\
     & -K_p^2 + (54-36a)K_p + 360 & 0 & 0
    \\\hline
    s^0 & 6K_p a & 0 & 0
    \\
     & K_p a & 0 & 0
    \\\hline
    \end{matrix}
  \]

    \item 
    \begin{enumerate}[I.]
      \item 
        \begin{align*}
          L(s) &= \frac{K_p}{s(s^2+s+1)(s+5)^2}
          \\
          p(s) &= s^5 + 11 s^4 + 36 s^3 + 35 s^2 + 25 s + K_p
        \end{align*}
        \[
          \begin{matrix}
          \hline
          s^5 & 1 & 36 & 25
          \\\hline
          s^4 & 11 & 35 & K_p
          \\\hline
          s^3 & \frac{361}{11} & \frac{275-K_p}{11} & 0
          \\
          & 361 & 275-K_p & 0
          \\\hline
          s^2 & \frac{9610-11K_p}{361} & K_p & 0
          \\
          & 9610-11K_p & 361 K_p & 0
          \\\hline
          s^1 & (9610-11K_p)(275+K_p) - 361^2K_p & 0 & 0
          \\
          & 2642750 - 123736 K_p + 3025K_p^2 & 0 & 0
          \\\hline
          s^0 & 361K_p & 0 & 0
          \\\hline
          \end{matrix}
        \]

        \begin{align*}
          361K_p > 0 \Rightarrow K_p > 0
          \\ \\
          2642750 - 123736 K_p + 3025K_p^2 > 0
          \\
          \Delta = (123736)^2-4\times3025\times 2642750 < 0
          \\
          \Rightarrow \text{No hay raices reales.}
          \\
          \Rightarrow \text{Como es cóncava hacia arriba, siempre es positivo.}
          \\
          \\
          9610-11K_p > 0 \Rightarrow K_p < \frac{9610}{11} = 873.6363... 
        \end{align*}
        El rango de $K_p$ para que el sistema sea estable es:
        \begin{align*}
          0 < K_p < 873.6363...
        \end{align*}
      \item 
        \begin{align*}
          L(s) &= \frac{200}{s(s^3+6s^2+11s+6)}
          \\
          p(s) &= s^4 + 6s^3 + 11s^2 + 6s + 200
        \end{align*}
        \[
          \begin{matrix}
          \hline
          s^4 & 1 & 11 & 200
          \\\hline
          s^3 & 6 & 6 & 0
          \\
           & 1 & 1 & 0 & \cdot / 6
          \\\hline
          s^2 & 10 & 200 & 0
          \\
          & 1 & 20 & 0 & \cdot / 10
          \\\hline
          s^1 & -19 & 0 & 0
          \\
           & -1 & 0 & 0 & \cdot / 19
          \\\hline
          s^0 & -20 & 0 & 0
          \\\hline
          \end{matrix}
        \]

        Se tiene un cambio de signo, por lo que se tienen dos polos conjugados en el semiplano derecho del plano complejo. Se tiene un sistema inestable.


      \item 
        \begin{align*}
          L(s) &= \frac{128}{s(s^7+3s^6+10s^5+24s^4+48s^3+96s^2+128s+192)}
          \\
          p(s) &= s^8 + 3s^7 + 10s^6 + 24s^5 + 48s^4 + 96s^3 + 128s^2 + 192s + 128
        \end{align*}
        \[
          \begin{matrix}
          \hline
          s^8 & 1 & 10 & 48 & 128 & 128
          \\\hline
          s^7 & 3 & 24 & 96 & 192 & 0 
          \\
           & 1 & 8 & 32 & 64 & 0 & \cdot / 3
          \\\hline
          s^6 & 2 & 16 & 64 & 128 & 0 
          \\
          & 1 & 8 & 32 & 64 & 0 & \cdot / 2
          \\\hline
          s^5 & 0 & 0 & 0 & 0 & 0  & \text{Caso especial}
          \\
           & 6\times1 & 8\times4 & 32\times2 & 0 & 0 & \partial (s^6 + 8 s^4 + 32 s^2 + 64) / \partial s
          \\
           & 3 & 16 & 32 & 0 & 0& \cdot / 2
          \\\hline
          s^4 & {^8}/_3 & {^{64}}/_3 & 64 & 0 & 0
          \\
           & 1 & 8 & 24 & 0 & 0
          \\\hline
          s^3 & -8 & -40 & 0 & 0 & 0 
          \\
           & -1 & -5 & 0 & 0 & 0 & \cdot / 8
          \\\hline
          s^2 & 3 & 24 & 0 & 0 & 0 
          \\
           & 1 & 8 & 0 & 0 & 0 & \cdot / 3
          \\\hline
          s^1 & 3 & 0 & 0 & 0 & 0
          \\\hline
          s^0 & 8 & 0 & 0 & 0 & 0
          \\\hline
          \end{matrix}
        \]

        Se tienen dos cambios de signo, por lo que se tienen dos polos en el semiplano derecho del plano complejo. Se tiene un caso especial con todos los elementos de la fila 0, por lo que se tiene un par de polos en el eje imaginario. Se tiene un sistema inestable.
        
    \end{enumerate}
    \item
    \begin{verbatim}
L =
                    1
  -------------------------------------
  s^5 + 11 s^4 + 36 s^3 + 35 s^2 + 25 s

sys =
    s^5 + 11 s^4 + 36 s^3 + 35 s^2 + 25 s
  -----------------------------------------
  s^5 + 11 s^4 + 36 s^3 + 35 s^2 + 25 s + 1

p =
  -5.0948 + 0.0000i
  -4.8992 + 0.0000i
  -0.4818 + 0.8441i
  -0.4818 - 0.8441i
  -0.0424 + 0.0000i

L =
             200
  --------------------------
  s^4 + 6 s^3 + 11 s^2 + 6 s

sys =
     s^4 + 6 s^3 + 11 s^2 + 6 s
  --------------------------------
  s^4 + 6 s^3 + 11 s^2 + 6 s + 200

p =
  -4.2760 + 2.5409i
  -4.2760 - 2.5409i
   1.2760 + 2.5409i
   1.2760 - 2.5409i

L =
                                 128
  -----------------------------------------------------------------
  s^8 + 3 s^7 + 10 s^6 + 24 s^5 + 48 s^4 + 96 s^3 + 128 s^2 + 192 s

sys =
     s^8 + 3 s^7 + 10 s^6 + 24 s^5 + 48 s^4 + 96 s^3 + 128 s^2 + 192 s
  -----------------------------------------------------------------------
  s^8 + 3 s^7 + 10 s^6 + 24 s^5 + 48 s^4 + 96 s^3 + 128 s^2 + 192 s + 128

p =
   1.0000 + 1.7321i
   1.0000 - 1.7321i
   0.0000 + 2.0000i
   0.0000 - 2.0000i
  -1.0000 + 1.7321i
  -1.0000 - 1.7321i
  -2.0000 + 0.0000i
  -1.0000 + 0.0000i
    \end{verbatim}
  \end{enumerate}
\end{ejercicio}