\textbf{Caso 4}

\textit{Se utilizan fórmulas y resultados relevantes obtenidos en el caso 1.}

\begin{align*}
  P(s) &= 
  \frac{s^2+4\,s+13}{s^3+20\,s^2+\frac{449\,s}{4}+195}
  &
  C(s) &= K_p
  \frac{s+1}{s}
  \\\\
  L(s) &= K_p
  \frac{s^3+5\,s^2+17\,s+13}{s^4+20\,s^3+\frac{449\,s^2}{4}+195\,s}
  &
  p(s) &= 
  s^4+20\,s^3+\frac{449\,s^2}{4}+195\,s + K_p(s^3+5\,s^2+17\,s+13)
\end{align*}
\begin{itemize}
  \item Regla 1: Simetría del LGR.
  \item Regla 2: Inicio y final del LGR. Inicia en polos($\omega = 0$) y termina en ceros($\omega \rightarrow \infty$).
  \begin{align*}
    \text{polos:} && 0 && -12 && -4 + j0.5 && -4 - j0.5
  \end{align*}
  \begin{align*}
    \text{ceros:} && -2 + j3 && -2 - j3 && -1
  \end{align*}
  \item Regla 3: Número de ramas del LGR. Cantidad de polos($n$) menos cantidad de ceros($m$). $n-m = 1$.
  \item Regla 4: LGR sobre el eje real. 
  De acuerdo a los polos existentes, hay secciones del LGR de 0 a -1 y  de -12 a $-\infty$.
  \item Regla 5: Ángulos de las asíntotas.
  \begin{align*}
    \alpha_0 &= \frac{2\times0 + 1}{1}  180\degree  = 180\degree
  \end{align*}
  \item Regla 6: Intersección de las asíntotas con el eje real. No aplica pues $m-n<2$.
  \item Regla 7: Centroide de las raices. . No aplica pues $m-n<2$.
  \item Regla 8: Puntos de salida o entrada al eje real.

  \begin{align*}
    \frac{dp_1(\sigma)}{d\sigma} p_2(\sigma) = p_1(\sigma) \frac{dp_2(\sigma)}{d\sigma}
    \\
    (s^3+5\,s^2+17\,s+13)(4s^3+60\,s^2+\frac{449\,s^2}{2}+195) =\\ (3s^2+10\,s^2+17\,s)(s^4+20\,s^3+\frac{449\,s^2}{4}+195\,s)
    \\
    4\,s^6+40\,s^5+155\,s^4+1368\,s^3+6853\,s^2+11674\,s+10140 = 0
  \end{align*}
  Los resultados para $s$ son:
  \begin{itemize}
    \item -7.7260
    \item -4.0658
    \item -1.0476 + 1.1218i
    \item -1.0476 - 1.1218i
    \item 1.9435 + 5.5207i
    \item 1.9435 - 5.5207i
  \end{itemize}
  De estos resultados y la regla 4, se observa que ningun punto funciona.
  \item Regla 9: Ángulos de entrada o salida al eje real. No aplica pues $p<2$
  \item Regla 10: Ángulo de partida o llegada de un polo complejo o a un cero complejo.
  
  Ángulo de partida del polo complejo:-4+j0.5
  \begin{equation*}
  68.0741\degree
  \end{equation*}
Ángulo de partida del polo complejo:-4-j0.5
\begin{equation*}
  291.9259\degree
  \end{equation*}
Ángulo de llegada del cero:-2+j3
  \begin{equation*}
  90.4196\degree
  \end{equation*}
Ángulo de llegada del cero:-2-j3
  \begin{equation*}
  269.5804\degree
  \end{equation*}

  \item Regla 11: Punto de cruce del eje imaginario.

  \begin{align*}
    p_{1\text{im}}(j\omega) p_{2\text{re}}(j\omega) &=  p_{1\text{re}}(j\omega) p_{2\text{im}}(j\omega)
  \end{align*}

  Ahora se utiliza la solución encontrada para el problema que se resuelve:
  \begin{align*}
    p_{1\text{im}} &=-\omega^2+17
    \\
    p_{1\text{re}} &=-5\omega^2+13
    \\
    p_{2\text{im}} &=-20\omega^2 -195
    \\
    p_{2\text{re}} &=\omega^4 - 112.25 \omega^2
  \end{align*}
  \begin{align*}
    (-\omega^2+17)(-20\omega^2 -195) = (-5\omega^2+13)(\omega^4 - 112.25 \omega^2)
  \end{align*}
  \begin{align*}
    \omega = \pm j1.8039 && 4.6991 \pm j2.4144 && 4.6991 \pm j2.4144
  \end{align*}
  Ningun valor funciona.

  No se teene ningn valor de $K_p$ tal que todas las raices se encuentren en el eje real, debido a las secciones que van de un polo complejo hasta un cero complejo sin cruzar el eje real.

\end{itemize}
