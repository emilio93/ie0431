\textbf{Caso 6}

\textit{Se utilizan fórmulas y resultados relevantes obtenidos en el caso 1.}

\begin{align*}
  L(s) &= K_p
  \frac{\frac{4\,s^3}{5}+\frac{26\,s^2}{5}+\frac{32\,s}{5}+2}{s^4}
  &
  p(s) &= 
  \frac{s^4+\frac{4\,s^3}{5}+\frac{26\,s^2}{5}+\frac{32\,s}{5}+2}{s^4}
\end{align*}
\begin{itemize}
  \item Regla 1: Simetría del LGR.
  \item Regla 2: Inicio y final del LGR. Inicia en polos($\omega = 0$) y termina en ceros($\omega \rightarrow \infty$).
  \begin{align*}
    \text{polos:} && 0 \qquad  0 \qquad  0 \qquad  0
  \end{align*}
  \begin{align*}
    \text{ceros:} && -5 \qquad -1 \qquad -0.5
  \end{align*}

  \item Regla 3: Número de ramas del LGR. Cantidad de polos($n$) menos cantidad de ceros($m$). $n-m = 1$.
  \item Regla 4: LGR sobre el eje real. 
  De acuerdo a los polos existentes, hay secciones del LGR de -0.5 a -1 y de -5 hasta $-\infty$.
  \item Regla 5: Ángulos de las asíntotas.
  \begin{align*}
    \alpha_0 &= \frac{2\times0 + 1}{1}  180\degree  = 180\degree
  \end{align*}
  \item Regla 6: Intersección de las asíntotas con el eje real. No aplica pues $m-n<2$.
  \item Regla 7: Centroide de las raices. No aplica pues $m-n<2$.
  \item Regla 8: Puntos de salida o entrada al eje real.

  \begin{align*}
    \frac{dp_1(\sigma)}{d\sigma} p_2(\sigma) = p_1(\sigma) \frac{dp_2(\sigma)}{d\sigma}
    \\
    (\frac{4\,s^3}{5}+\frac{26\,s^2}{5}+\frac{32\,s}{5}+2)(4s^3) = (\frac{12\,s^2}{5}+\frac{52\,s}{5}+\frac{32}{5})(s^4)
    \\
    \left(s^3+13\,s^2+24\,s+10\right) = 0
  \end{align*}
  Los resultados para $s$ son:
  \begin{itemize}
    \item -0.6068
    \item -10.8783
    \item -1.5149
  \end{itemize}
  De estos resultados y la regla 4, se tiene que solo el punto $s=-0.6068$ y $s=-10.8783$ son válidos.
  \item Regla 9: Ángulos de entrada o salida al eje real.
  \begin{align*}
    \alpha_{c,0} = \frac{(2\times0+1)180\degree}{2} = 90\degree
    \\
    \alpha_{c,1} = \frac{(2\times1+1)180\degree}{2} = 270\degree = -90\degree
    \\
    \alpha_{c,0} = \frac{(2\times0+1)180\degree}{2} = 90\degree
    \\
    \alpha_{c,1} = \frac{(2\times1+1)180\degree}{2} = 270\degree = -90\degree
  \end{align*}
  \item Regla 10: Ángulo de partida o llegada de un polo complejo o a un cero complejo.
  No hay polos ni ceros complejos.

  \item Regla 11: Punto de cruce del eje imaginario.

  \begin{align*}
    p_{1\text{im}}(j\omega) p_{2\text{re}}(j\omega) &=  p_{1\text{re}}(j\omega) p_{2\text{im}}(j\omega)
  \end{align*}

  Ahora se utiliza la solución encontrada para el problema que se resuelve:
  \begin{align*}
    p_{1\text{im}} &=-0.8\omega^2+6.4
    \\
    p_{1\text{re}} &=-5.2\omega^2+2
    \\
    p_{2\text{im}} &=0
    \\
    p_{2\text{re}} &=\omega^4
  \end{align*}
  \begin{align*}
    (-0.8\omega^2+6.4)(\omega^4) = 0
    \\
    \omega = \pm 2\sqrt{2} =\pm2.8284
  \end{align*}
  Funcionan ambas soluciones.

   \item Regla 12:

  Las ganancias en las que entra o sale el LRG del eje real son $29$ y $0.9$.

\end{itemize}
