\textbf{Caso 3}

\textit{Se utilizan fórmulas y resultados relevantes obtenidos en el caso 1.}

\begin{align*}
  P(s) &= 
  \frac{10}{s^3+5\,s^2+12\,s+8}
  &
  C(s) &= K_p
  \frac{s^2+10\,s+34}{10\,s}
  \\\\
  L(s) &= K_p
  \frac{10\,s^2+100\,s+340}{10\,s^4+50\,s^3+120\,s^2+80\,s}
  &
  p(s) &= 
  10\,s^4+50\,s^3+120\,s^2+80\,s + K_p(10\,s^2+100\,s+340)
\end{align*}
\begin{itemize}
  \item Regla 1: Simetría del LGR.
  \item Regla 2: Inicio y final del LGR. Inicia en polos($\omega = 0$) y termina en ceros($\omega \rightarrow \infty$).
  \begin{align*}
    \text{polos:} && 0 \qquad  -2 + j2 \qquad  -2 - j2 \qquad  -1
  \end{align*}
  \begin{align*}
    \text{ceros:} && -0.5 + j3 \qquad -0.5 - j3
  \end{align*}

  \item Regla 3: Número de ramas del LGR. Cantidad de polos($n$) menos cantidad de ceros($m$). $n-m = 2$.
  \item Regla 4: LGR sobre el eje real. 
  De acuerdo a los polos existentes, hay secciones del LGR de 0 a -1.
  \item Regla 5: Ángulos de las asíntotas.
  \begin{align*}
    \alpha_0 &= \frac{2\times0 + 1}{2}  180\degree  = 90\degree
    \\
    \alpha_1 &= \frac{2\times1 + 1}{2}  180\degree = -90\degree
  \end{align*}
  \item Regla 6: Intersección de las asíntotas con el eje real.
  \begin{align*}
    \sigma_a &= 2.5
  \end{align*}
  \item Regla 7: Centroide de las raices.
  \begin{align*}
    \sigma_r &= -1.25
  \end{align*}
  \item Regla 8: Puntos de salida o entrada al eje real.

  \begin{align*}
    \frac{dp_1(\sigma)}{d\sigma} p_2(\sigma) = p_1(\sigma) \frac{dp_2(\sigma)}{d\sigma}
    \\
    (40s^3+150s^2)(10s^4+50s^3+120s^2+80s) = (40s^3+150s^2+240s+80)(10s^4+50s^3)
    \\
    2\,s^5+35\,s^4+236\,s^3+622\,s^2+816\,s+272 = 0
  \end{align*}
  Los resultados para $s$ son:
  \begin{itemize}
    \item -0.4782
    \item -6.9282 + j3.6557
    \item -6.9282 - j3.6557
    \item -1.5827 + j1.4593
    \item -1.5827 - j1.4593
  \end{itemize}
  De estos resultados y la regla 4, se tiene que solo el punto $s=-0.4782$ es válido.
  \item Regla 9: Ángulos de entrada o salida al eje real.
  \begin{align*}
    \alpha_{c,0} = \frac{(2\times0+1)180\degree}{2} = 90\degree
    \\
    \alpha_{c,1} = \frac{(2\times1+1)180\degree}{2} = 270\degree = -90\degree
  \end{align*}
  \item Regla 10: Ángulo de partida o llegada de un polo complejo o a un cero complejo.

  Ángulo de partida del polo complejo:-2+2i
  \begin{equation*}
  202.1663\degree
  \end{equation*}
  Ángulo de partida del polo complejo:-2-2i
  \begin{equation*}
  157.8337\degree
  \end{equation*}
  Ángulo de llegada del cero:-5+3i
  \begin{equation*}
  124.6952\degree
  \end{equation*}
  Ángulo de llegada del cero:-5-3i
  \begin{equation*}
  235.3048\degree
  \end{equation*}

  \item Regla 11: Punto de cruce del eje imaginario.

  \begin{align*}
    p_{1\text{im}}(j\omega) p_{2\text{re}}(j\omega) &=  p_{1\text{re}}(j\omega) p_{2\text{im}}(j\omega)
  \end{align*}

  Ahora se utiliza la solución encontrada para el problema que se resuelve:
  \begin{align*}
    p_{1\text{im}} &=100
    \\
    p_{1\text{re}} &=-10\omega^2+340
    \\
    p_{2\text{im}} &=-50\omega^2 -80
    \\
    p_{2\text{re}} &=10\omega^4 - 120 \omega^2
  \end{align*}
  \begin{align*}
    (100)(-50\omega^2 -80) = (-10\omega^2+340)(10\omega^4 - 120 \omega^2)
    \\
    \omega = \pm j3.8965 \qquad, \qquad \pm 1.8929
  \end{align*}
  Funciona $1.89229$.

  \item Regla 12: Cálculo de la ganancia en un punto LGR:
  
  Para $\pm1.8929$:
    \begin{equation*}
    |K|_{s=1.8929} = 0.99
  \end{equation*}

\end{itemize}
