\textbf{Caso 5}

\textit{Se utilizan fórmulas y resultados relevantes obtenidos en el caso 1.}

\begin{align*}
  P(s) &= 
  \frac{s^2+4\,s+13}{s^3+20\,s^2+\frac{449\,s}{4}+195}
  &
  C(s) &= K_p
  \frac{\frac{3\,s^2}{2}+3\,s+4}{3\,s}
  \\\\
  L(s) &= K_p
  \frac{\frac{3\,s^4}{2}+9\,s^3+\frac{71\,s^2}{2}+55\,s+52}{3\,s^4+60\,s^3+\frac{1347\,s^2}{4}+585\,s}
  \\\\
  p(s) &= 
  3\,s^4+60\,s^3+\frac{1347\,s^2}{4}+585\,s + K_p(\frac{3\,s^4}{2}+9\,s^3+\frac{71\,s^2}{2}+55\,s+52)
\end{align*}
\begin{itemize}
  \item Regla 1: Simetría del LGR.
  \item Regla 2: Inicio y final del LGR. Inicia en polos($\omega = 0$) y termina en ceros($\omega \rightarrow \infty$).
  \begin{align*}
    \text{polos:} && 0 \qquad  -12 \qquad  -4 + j0.5 \qquad  -4 - j0.5
  \end{align*}
  \begin{align*}
    \text{ceros:} && -2 + j3 \qquad -2 - j3 \qquad -1 + j1.2910 \qquad -1 - j1.2910
  \end{align*}

  \item Regla 3: Número de ramas del LGR. Cantidad de polos($n$) menos cantidad de ceros($m$). $n-m = 0$.
  \item Regla 4: LGR sobre el eje real. 
  De acuerdo a los polos existentes, hay secciones del LGR de 0 a -12.
  \item Regla 5: Ángulos de las asíntotas.
  \begin{align*}
    \alpha_0 &= \frac{2\times0 + 1}{2}  180\degree  = 180\degree
  \end{align*}
  \item Regla 6: Intersección de las asíntotas con el eje real. No aplica pues $m-n<2$.
  \item Regla 7: Centroide de las raices. No aplica pues $m-n<2$.
  \item Regla 8: Puntos de salida o entrada al eje real.

  \begin{align*}
    \frac{dp_1(\sigma)}{d\sigma} p_2(\sigma) = p_1(\sigma) \frac{dp_2(\sigma)}{d\sigma}
    \\
    (6\,s^3+27\,s^2+71\,s+55)(3\,s^4+60\,s^3+\frac{1347\,s^2}{4}+585\,s) = \\(\frac{3\,s^4}{2}+9\,s^3+\frac{71\,s^2}{2}+55\,s+52)(12\,s^3+180\,s^2+\frac{1347\,s}{2}+585)
    \\
    -252\,s^6-3189\,s^5-12153\,s^4-13224\,s^3+28455\,s^2+140088\,s+121680 = 0
  \end{align*}
  Los resultados para $s$ son:
  \begin{itemize}
    \item -6.8911
    \item -4.0967
    \item -1.1911
    \item 2.1400
    \item -1.3079 + 2.2360i
    \item -1.3079 - 2.2360i
  \end{itemize}
  De estos resultados y la regla 4, se tiene que los punto $s=-6.8911$, $s=-4.0967$, $s=-1.1911$ son válido.
  \item Regla 9: Ángulos de entrada o salida al eje real.
  \begin{align*}
    \alpha_{c,0} = \frac{(2\times0+1)180\degree}{2} = 90\degree
    \\
    \alpha_{c,1} = \frac{(2\times1+1)180\degree}{2} = 270\degree = -90\degree
  \end{align*}
  \item Regla 10: Ángulo de partida o llegada de un polo complejo o a un cero complejo.

  Ángulo de partida del polo complejo:-4+j0.5
  \begin{equation*}
  61.47\degree
  \end{equation*}
  Ángulo de partida del polo complejo:-4-j0.5
  \begin{equation*}
  298.53\degree
  \end{equation*}
  Ángulo de llegada del cero:-2+j3
  \begin{equation*}
  25.4365\degree
  \end{equation*}
  Ángulo de llegada del cero:-2-j3
  \begin{equation*}
  334.5635\degree
  \end{equation*}
  Ángulo de llegada del cero:-1+j1.291
  \begin{equation*}
  107.2779\degree
  \end{equation*}
  Ángulo de llegada del cero:-1-j1.291
  \begin{equation*}
  252.7221\degree
  \end{equation*}

  \item Regla 11: Punto de cruce del eje imaginario.

  \begin{align*}
    p_{1\text{im}}(j\omega) p_{2\text{re}}(j\omega) &=  p_{1\text{re}}(j\omega) p_{2\text{im}}(j\omega)
  \end{align*}

  Ahora se utiliza la solución encontrada para el problema que se resuelve:
  \begin{align*}
    p_{1\text{im}} &=-9\omega^2+55
    \\
    p_{1\text{re}} &=1.5\omega^4-35.5\omega^2+52
    \\
    p_{2\text{im}} &=-60\omega^2 -585
    \\
    p_{2\text{re}} &=3\omega^4 - 336.75 \omega^2
  \end{align*}
  \begin{align*}
    (-9\omega^2+55)(-60\omega^2 -585) = (1.5\omega^4-35.5\omega^2+52)(3\omega^4 - 336.75 \omega^2)
    \\
    \omega = \pm 2.0469 \qquad, \qquad 1.8903 \pm j2.6762 \qquad -1.8903 \pm j2.6762
  \end{align*}
  Funciona $\pm2.0469$.

  Para que todos los polos tengan parte real únicamente se debe cumplir que $0.5<K_p<12.7$.

\end{itemize}
