\begin{ejercicio}

\begin{itemize}
  \item 
  Se utilizan los parámetros para la equivalencia de los parámetros. Primeramente para pasar de PID en serie de 2GdL a uno estandard.
  \begin{align*}
    F_C^{'} &= 1 + \frac{(1-\alpha^{'})T_d^{'}}{T_i^{'}}
    &
    K_p &= F_C^{'} K_P^{'} 
    &
    T_i &= F_C^{'} T_i^{'}
    \\
    T_d &= \frac{(1-\alpha^{'} F_C^{'}) T_d^{'}}{F_C^{'}}
    &
    \alpha &= \frac{F_C^{'}  \alpha^{'}}{1-F_C^{'}  \alpha^{'}}, \alpha^{'} <1+\frac{T_i^{'}}{T_d^{'}}
    &
    \beta &= \frac{\beta^{'}}{F_C^{'}}
  \end{align*}

  Para los valores del enunciado se tienen los parámetros equivalentes para un controlador PID estandard de 2GdL:
  \begin{align*}
    F_C^{'} &= 1 + \frac{0.6(1-0.1)}{1.9} = \frac{122}{95} \approx 1.2842
    &
    K_p &= \frac{122}{95}\times 2.3 = \frac{1403}{475} \approx 2.9537
    \\\\
    T_i &= \frac{122}{95}\times 1.9 = \frac{61}{25} = 2.44
    &
    T_d &= \frac{0.6(1-0.1 \times \frac{122}{95})}{\frac{122}{95}} = \frac{621}{1525} \approx 0.4072
    \\\\
    \alpha &= \frac{\frac{122}{95} \times 0.1}{1-\frac{122}{95} \times 0.1} = \frac{61}{414} \approx 0.1473
    &
    \beta &= \frac{1.1}{\frac{122}{95}} = \frac{209}{244} = 0.8566
  \end{align*}

  Utilizando los parámetros del PID estandard de 2GdL se puede pasar a  los parámetros para un PID en paralelo de 2GdL:
  \begin{align*}
    K_P &= K_P
    &
    K_i &= \frac{K_P}{T_I}
    &
    K_d &= K_P T_d 
    \\
    \alpha_P &= \frac{\alpha}{K_P}
    &
    \beta_P &= \beta
  \end{align*}

  Para los valores del PID estandard de 2GdL obtenidos se tienen los parámetros equivalentes para un controlador PID en paralelo de 2GdL:
    \begin{align*}
    K_P &= \frac{1403}{475} \approx 2.9537
    &
    K_i &= \frac{\frac{1403}{475}}{2.44} = \frac{23}{19} \approx 1.2105
    \\
    K_d &= \frac{1403}{475} \times \frac{621}{1525} \approx 1.2028
    &
    \alpha_P &= \frac{\frac{61}{414}}{\frac{1403}{475}} =\frac{475}{9522} \approx 0.0499
    \\
    \beta_P &= \frac{209}{244} = 0.8566
  \end{align*}

  \item
  Utilizando los parámetros del PID en paralelo de 2GdL se puede pasar a los parámetros para un PID estandard de 2GdL:
  \begin{align*}
    K_P &= K_P
    &
    T_i &= \frac{K_P}{K_i}
    &
    T_d &= \frac{K_d}{K_P}
    \\
    \alpha &= \aplpha_P K_P
    &
    \beta &= \beta_P
  \end{align*}

  Para los valores del PID en paralelo de 2GdL obtenidos se tienen los parámetros equivalentes para un controlador PID estandard de 2GdL:
  \begin{align*}
    K_P &= \frac{1403}{475} \approx 2.9537
    &
    T_i &= \frac{\frac{1403}{475}}{\frac{23}{19}} = \frac{61}{25} = 2.44
    \\
    T_d &= \frac{\frac{621}{1525}}{\frac{1403}{475}} = \frac{621}{1525} \approx 0.4072
    &
    \alpha &= \frac{475}{9522} \times \frac{1403}{475} = \frac{61}{414} \approx 0.1473
    \\
    \beta &= \frac{209}{244} = 0.8566
  \end{align*}

  Finalmente se utilizan las equivalencias para pasar parámetros de un PID estandard de 2GdL a uno en serie.
  \begin{align*}
    F_C &= \frac{1}{2}\left[ 1+ \frac{\alpha T_d}{T_i} + \sqrt{1-\frac{(4+2\alpha)T_d}{T_i} + \frac{\alpha^2T_d^2}{T_i^2}} \right]
    &
    K_P^{'} &= F_C K_P
    &
    T_i^{'} &= F_C T_i
    \\
    T_d^{'} &= \frac{(1+\alpha)T_d}{F_C}
    &
    \alpha^{'} &= \frac{\alpha F_C}{1+\alpha}
    &
    \beta^{'} &= \frac{\beta}{F_C}
  \end{align*}

  Se sustituyen los valores para obtener los parámetros del PID en serie de 2GdL correspondiente al PID estandard obtenido previamente.
  \begin{align*}
    F_C &= \frac{1}{2}\left[ 1+ \frac{\frac{61}{414}\times \frac{621}{1525}}{2.44} + \sqrt{1-\frac{(4+2\frac{61}{414})\times \frac{621}{1525}}{2.44} + \frac{\left(\frac{61}{414}\right)^2\left(\frac{621}{1525}\right)^2}{2.44^2}} \right] = \frac{95}{122} \approx 0.7787
  \end{align*}
  \begin{align*}
    K_P^{'} &= \frac{95}{122} \times \frac{1403}{475} = 2.3
    &
    T_i^{'} &= \frac{95}{122} \times 2.44 = 1.9
    \\
    T_d^{'} &= \frac{(1+\frac{61}{414})\frac{621}{1525}}{\frac{95}{122}} = 0.6
    &
    \alpha^{'} &= \frac{\frac{61}{414} \times \frac{95}{122}}{1+\frac{61}{414}} = 0.1
    \\
    \beta^{'} &= \frac{\frac{209}{244}}{\frac{95}{122}} = 1.1
  \end{align*}
\end{itemize}

\end{ejercicio}