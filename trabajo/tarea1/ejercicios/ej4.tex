\begin{ejercicio}
  Se utilizan los parámetros para la equivalencia de los parámetros. Primeramente para pasar de PID en serie de 2GdL a uno estandard.

  \begin{align*}
    F_C^{'} &= 1 + \frac{(1-\alpha^{'})T_d^{'}}{T_i^{'}}
    &
    K_p &= F_C^{'} K_P^{'} 
    &
    T_i &= F_C^{'} T_i^{'}
    \\
    T_d &= \frac{(1-\alpha^{'} F_C^{'}) T_d^{'}}{F_C^{'}}
    &
    \alpha &= \frac{F_C^{'}  \alpha^{'}}{1-F_C^{'}  \alpha^{'}}, \alpha^{'} <1+\frac{T_i^{'}}{T_d^{'}}
    &
    \beta &= \frac{\beta^{'}}{F_C^{'}}
  \end{align*}

Para los valores del enunciado se tienen los parámetros equivalentes para un controlador PID estandard de 2GdL:

  \begin{align*}
    F_C^{'} &= 1 + \frac{0.6(1-0.1)}{1.9} = \frac{122}{95} \approx 1.2842
    &
    K_p &= \frac{122}{95}\times 2.3 = \frac{1403}{475} \approx 1.9537
    \\\\
    T_i &= \frac{122}{95}\times 1.9 = \frac{61}{25} = 2.44
    &
    T_d &= \frac{0.6(1-0.1 \times \frac{122}{95})}{\frac{122}{95}} = \frac{621}{1525} \approx 0.4072
    \\\\
    \alpha &= \frac{\frac{122}{95} \times 0.1}{1-\frac{122}{95} \times 0.1} = \frac{61}{414} \approx 0.1473
    &
    \beta &= \frac{1.1}{\frac{122}{95}} = \frac{19}{4} = 4.75
  \end{align*}

Utilizando los parámetros del PID estandard de 2GdL se puede pasar a los parámetros para un PID en paralelo de 2GdL:

  \begin{align*}
    F_C^{'} &= 1 + \frac{(1-\alpha^{'})T_d^{'}}{T_i^{'}}
    &
    K_p &= F_C^{'} K_P^{'} 
    &
    T_i &= F_C^{'} T_i^{'}
    \\
    T_d &= \frac{(1-\alpha^{'} F_C^{'}) T_d^{'}}{F_C^{'}}
    &
    \alpha &= \frac{F_C^{'}  \alpha^{'}}{1-F_C^{'}  \alpha^{'}}, \alpha^{'} <1+\frac{T_i^{'}}{T_d^{'}}
    &
    \beta &= \frac{\beta^{'}}{F_C^{'}}
  \end{align*}

\end{ejercicio}