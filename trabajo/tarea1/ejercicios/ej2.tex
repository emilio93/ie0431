  \begin{ejercicio}
    \begin{figure}[H]
        \centering
        \includegraphics{tarea1/tikz/control-realimentado.tikz}
        \caption{Sistema de control realimentado simple.}
        \label{img:ejercicio1}
    \end{figure}
  \end{ejercicio}
  \begin{itemize}
    \item 
    Se obtiene la señal de salida $y(s)$ en términos de las entradas $r(s)$ y $d(s)$.
    \begin{align*}
      y(s) &= P(s) \left[ d(s) + C(s) \left( r(s) - y(s) \right) \right]
      \\
      y(s) \left[ 1 + C(s) P(s) \right] &= P(s) \left[ d(s) + r(s) C(s) \right]
      \\
      y(s) &= \frac{P(s) \left[ d(s) + r(s) C(s) \right]}{1 + C(s) P(s)}
    \end{align*}

    \item
    Se obtiene la señal de control $u(s)$ en términos de las entradas $r(s)$ y $d(s)$.
    \begin{align*}
      u(s) &= C(s) \left[ r(s) - P(s) \left[ u(s) + d(s) \right] \right]
      \\
      u(s) \left[ 1 + C(s) P(s) \right] &= C(s) \left[ r(s) - P(s) d(s) \right]
      \\ \\
      u(s) &= \frac{C(s) \left[ r(s) - P(s) d(s) \right]}{1 + C(s) P(s)}
    \end{align*}

    \item
    Se obtiene la señal de error $e(s)$ en términos de las entradas $r(s)$ y $d(s)$.
    \begin{align*}
      e(s) &= r(s) - P(s) \left[ d(s) + e(s) C(s) \right]
      \\
      e(s) \left[ 1 + C(s) P(s) \right] &= r(s) - P(s) d(s)
      \\ \\
      e(s) &= \frac{r(s) - P(s) d(s)}{1 + C(s) P(s)}
    \end{align*}

    \item
    \begin{itemize}

      \item 
      Se obtiene la función de transferencia de servo-control:
      \begin{align*}
        y_{r}(s) &= y(s)|_{d(s)=0} = \frac{P(s) C(s) r(s)}{1 + C(s) P(s)}
      \end{align*}

      \begin{equation*}
        \tag{Función de Transferencia del servo control}
        M_{yr}(s) = \frac{y_r(s)}{r(s)} = \frac{C(s) P(s)}{1 + C(s) P(s)}
      \end{equation*}

      \item
      Se obtiene la función de transferencia de control regulatorio:
      \begin{align*}
        y_{d}(s) &= y(s)|_{r(s)=0} = \frac{P(s) d(s) }{1 + C(s) P(s)}
      \end{align*}

      \begin{equation*}
        \tag{Función de Transferencia del control regulatorio}
        M_{yd}(s) = \frac{y_d(s)}{d(s)} = \frac{P(s)}{1 + C(s) P(s)}
      \end{equation*}

      \item
      Se obtiene la función de sensibilidad:
      \begin{align*}
        S_P^{M_{yr}}(s) &= \frac{P}{M_{yr}} \frac{\partial M_{yr}}{\partial P} 
          =
          \frac{P(s)}{\frac{C(s) P(s)}{1 + C(s) P(s)}} 
          \frac{\partial \frac{C(s) P(s)}{1 + C(s) P(s)}}{\partial P} 
          \\
        S_P^{M_{yr}}(s) &=
          \frac{P(s)(1 + C(s) P(s))}{C(s) P(s)} \cdot
          \frac{C(s)(1 + C(s) P(s)) - C^2(s)P(s)}{(1 + C(s) P(s))^2}
          \\
        S_P^{M_{yr}}(s) &=
          \frac{P(s)}{C(s) P(s)} \cdot
          \frac{C(s)\left[(1 + C(s) P(s)) - C(s)P(s)\right]}{1 + C(s) P(s)}
          \\
        S_P^{M_{yr}}(s) &=
          \frac{(1 + C(s) P(s)) - C(s)P(s)}{1 + C(s) P(s)} =
          \frac{1}{1 + C(s) P(s)}
      \end{align*}

      \item
      La función de sensibilidad complementaria:
      \begin{align*}
        T(s) &= 1 - S_P^{M_{yr}}(s)
        \\
        T(s) &= 1 - \frac{1}{1 + C(s) P(s)} = \frac{1-(1+C(s)P(s))}{1 + C(s) P(s)} = \frac{C(s) P(s)}{1 + C(s) P(s)}
      \end{align*}
    \end{itemize}
  \end{itemize}



