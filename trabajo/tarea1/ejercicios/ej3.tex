\begin{ejercicio}
  \begin{itemize}
    \item 
    Se tiene un fluido refrigerante, por lo que cuando la válvula está completamente abierta, se asume que el sistema baja su temperatura.\\\par

    \textbf{Caso NA}\par
    Cuando la temperatura($y(s)$) aumenta sobre su valor de referencia, el control debe encargarse de enfriar el sistema, esto es aumentar el flujo $m(s)$, por lo que envia una señal para aumentar el flujo del refrigerante, lo que es disminuir $u(s)$, abriendo la válvula y por ende aumentando el fluido, como es deseado, se tiene entonces un proceso de acción inversa (+1).

    \begin{equation*}
      y(s) \mspace{6mu} \nearrow  \quad \Rightarrow \quad  m(s) \mspace{6mu} \nearrow  \quad \Rightarrow \quad  u(s) \searrow \quad \Rightarrow \quad \text{Acción Inversa (+1)}
    \end{equation*}

    \textbf{Caso NC}\par
    Cuando la temperatura($y(s)$) aumenta sobre su valor de referencia, el control debe encargarse de enfriar el sistema, esto es aumentar el flujo $m(s)$, por lo que envia una señal para aumentar el flujo del refrigerante, lo que es aumentar $u(s)$, abriendo la válvula y por ende aumentando el fluido, como es deseado, se tiene entonces un proceso de acción directa (-1).

    \begin{equation*}
      y(s) \mspace{6mu} \nearrow  \quad \Rightarrow \quad  m(s) \mspace{6mu} \nearrow  \quad \Rightarrow \quad  u(s) \nearrow \quad \Rightarrow \quad \text{Acción Directa (-1)}
    \end{equation*}

  \item
  En la figura \ref{ej3:diag1} se muestra el diagrama de bloques del sistema analizado. El recuadro rojo muestra el controlador, mientras que el cuadro verde muestra el proceso controlado. La señal resultante de la resta $r(s) - y(s)$ es la señal de error $e(s)$. La señal resultante del controlador $u(s)$ se encuentra normalizada, por lo que es multiplicada por $K_{VC}$ para obtener el valor deseado de la variable manipulada $m(s)$, el flujo del refrigerante, este pasa a través del proceso, al sistema se le mide la temperatura $T(s)$, y se normaliza con el factor $K_T$, para obtener la señal normalizada de retroalimentación.
  \begin{figure}[H]
      \centering
      \includegraphics{tarea1/tikz/control-realimentado-termico.tikz}
      \caption{Diagrama del sistema de refrigeración.}
      \label{ej3:diag1}
    \end{figure}
  \end{itemize}
\end{ejercicio}