\begin{ejercicio}
  \begin{itemize}
    \item 
    Se tiene un fluido refrigerante, por lo que cuando la válvula está completamente abierta, se asume que el sistema baja su temperatura.\\\par

    \textbf{Caso NA}\par
    Cuando la temperatura($y(s)$) aumenta sobre su valor de referencia, el control debe encargarse de enfriar el sistema, esto es aumentar el flujo $m(s)$, por lo que envia una señal para aumentar el flujo del refrigerante, lo que es disminuir $u(s)$, abriendo la válvula y por ende aumentando el fluido, como es deseado, se tiene entonces un proceso de acción inversa (+1).

    \begin{equation*}
      y(s) \mspace{6mu} \nearrow  \quad \Rightarrow \quad  m(s) \mspace{6mu} \nearrow  \quad \Rightarrow \quad  u(s) \searrow \quad \Rightarrow \quad \text{Acción Inversa (+1)}
    \end{equation*}

    \textbf{Caso NC}\par
    Cuando la temperatura($y(s)$) aumenta sobre su valor de referencia, el control debe encargarse de enfriar el sistema, esto es aumentar el flujo $m(s)$, por lo que envia una señal para aumentar el flujo del refrigerante, lo que es aumentar $u(s)$, abriendo la válvula y por ende aumentando el fluido, como es deseado, se tiene entonces un proceso de acción directa (-1).

    \begin{equation*}
      y(s) \mspace{6mu} \nearrow  \quad \Rightarrow \quad  m(s) \mspace{6mu} \nearrow  \quad \Rightarrow \quad  u(s) \nearrow \quad \Rightarrow \quad \text{Acción Directa (-1)}
    \end{equation*}

  \item
  Diagrama
  \end{itemize}
\end{ejercicio}