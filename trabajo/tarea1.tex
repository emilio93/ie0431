\documentclass{ucrEieTarea}

\usepackage{sansmathfonts}
\usepackage{environ}
\usepackage{varwidth}

\graphicspath{{./trabajo/}}

\renewcommand*\familydefault{\sfdefault} %% Only if the base font of the document is to be sans serif
    
\begin{document}
  \portadaminima{Universidad de Costa Rica}
    {Escuela de Ingeniería Eléctrica}
    {IE0431 - Sistemas de Control}
    {Emilio Javier Rojas Álvarez}
    {B15680}
    {Grupo 901}
    {Tarea \#1}
  
  \begin{ejercicio}

  La ganancia del transmisor se obtiene normalizada de 0 a 1, se hace respecto al ámbito de medición, $1/3.8m = 5/19$, lo que es un $26.3\%$.
  
  A partir de la característica estática se obtiene el valor de $Q_e$ en términos de $H$ y $X_{VS}$:

  \begin{align*}
    H &=\frac{1}{\rho g} \left( \frac{Q_e}{X_{VS}K_{VS}} \right)
    \\
    Q_e &= X_{VS} K_{VS}\sqrt{\rho g H}
  \end{align*}

  Con esto se averigüa los valores máximo y mínimo para el caudal de entrada según los valores en el enunciado según corresponda.

  \begin{align*}
    Q_{emin} &= 0.4 \times 0.001 \times \sqrt{1475\times9.81\times2.5} \approx 0.0761
    \\
    Q_{emax} &= 0.5 \times 0.001 \times \sqrt{1475\times9.81\times3.5} \approx 0.1350
  \end{align*}





    \begin{figure}[H]
      \centering
      \includegraphics[width=0.7\textwidth]{tarea1/ej1-caracteristicaestatica.eps}
    \end{figure}

        \begin{figure}[H]
      \centering
      \includegraphics[width=0.7\textwidth]{tarea1/ej1-caracteristicaestaticazoom.eps}
    \end{figure}
  \end{ejercicio}
  
  \begin{ejercicio}
    \begin{figure}[H]
        \centering
        \includegraphics{tarea1/control-realimentado.tikz}
        \caption{Sistema de control realimentado simple.}
        \label{img:ejercicio1}
    \end{figure}
  \end{ejercicio}

  Se obtiene la señal de salida $y(s)$ en términos de las entradas $r(s)$ y $d(s)$.
  \begin{align*}
    y(s) &= P(s) \left[ d(s) + C(s) \left( r(s) - y(s) \right) \right]
    \\
    y(s) \left[ 1 + C(s) P(s) \right] &= P(s) \left[ d(s) + r(s) C(s) \right]
    \\ \\
    y(s) &= \frac{P(s) \left[ d(s) + r(s) C(s) \right]}{1 + C(s) P(s)}
  \end{align*}

  Se obtiene la señal de control $u(s)$ en términos de las entradas $r(s)$ y $d(s)$.
  \begin{align*}
    u(s) &= C(s) \left[ r(s) - P(s) \left[ u(s) + d(s) \right] \right]
    \\
    u(s) \left[ 1 + C(s) P(s) \right] &= C(s) \left[ r(s) - P(s) d(s) \right]
    \\ \\
    u(s) &= \frac{C(s) \left[ r(s) - P(s) d(s) \right]}{1 + C(s) P(s)}
  \end{align*}

  Se obtiene la señal de error $e(s)$ en términos de las entradas $r(s)$ y $d(s)$.
  \begin{align*}
    e(s) &= r(s) - P(s) \left[ d(s) + e(s) C(s) \right]
    \\
    e(s) \left[ 1 + C(s) P(s) \right] &= r(s) - P(s) d(s)
    \\ \\
    e(s) &= \frac{r(s) - P(s) d(s)}{1 + C(s) P(s)}
  \end{align*}


  \begin{ejercicio}

  \end{ejercicio}

  \begin{ejercicio}

  \end{ejercicio}

  \begin{ejercicio}

  \end{ejercicio}


\end{document}